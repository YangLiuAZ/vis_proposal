\section{Background}
\label{sec:background}

There're many existing approaches to compare two entities. The comparison can be user-based, that is, comparing similarity of two users; or item-based, that is, comparing similarity of two objects. Item-based comparison techniques are more close to our need. Well known metrics include Euclidean distance, Jaccard similarity, cosine similarity etc.

\subsection{Related Work}
\label{sec:related}

The problem of coursework similarity has been studied in the context of course recommendation system. Bendakir et al.~\cite{bendakir2006using} proposed a recommendation system based on decision tree of course history. Their approach, however, does not consider students' grades at all. Thus, their tool may wrongly correlate totally different courses simply due to historical mistakes. Sandvig et al.~\cite{sandvig2005aacorn} did use the GPA information, but GPA, as an average metric, doesn't say much about each specific class. 

When it comes to the visualization problem. Since our goal is to cluster similar classes together, a node-link diagram naturally jumps into our mind. D3 library has a force-directed graph that is close to our needs. But we are hesitant about its fisheye distortion and curved link variant because these variants make it hard to click on nodes or edges for further details. We are also aware that force directed drawing is criticized for local minima. A multilevel approach~\cite{walshaw2000multilevel} might fix it but we are not focusing on algorithmic style improvement in this proposal.

